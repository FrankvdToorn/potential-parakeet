\chapter{Fieldwork data analysis}
Geological conditions are highly heterogeneous in northern Ghana. Subsurface characteristics vary at short mutual distances. Adequate and reliable information about local geohydrological conditions is preferably gathered through site-specific fieldwork. In this research perspective, multiple northern Ghana borehole locations are subjected to groundwater pumping tests. 
\bigskip \\
The NGO Conservation Alliance (CA) installed several PIT locations in the summer of 2016, in the Upper East and Northern Region. Pumping tests are performed at four of these boreholes. A fifth PIT borehole (in Ziong) is monitored to study how the ASR system is used by local farmers. The figure below shows a map  of the research locations in northern Ghana (Figure~\ref{fig:Overviewlocations}).
\bigskip 

\begin{figure}[ht]
 \centering
 \includegraphics[width=\linewidth]{Overview_locations_northern_Ghana}
 \captionsetup{justification=centering} 
 \caption{Overview of fieldwork locations in northern Ghana}
 \label{fig:Overviewlocations}
\end{figure}

Detailed information on the equipment that was used, the set-up of the pumping tests as well as the monitoring of an operating ASR system can be found in appendix \ref{chapter:fieldwork_set-up}. The obtained raw fieldwork data can be found in the site-specific fact-sheets of appendix \ref{section:fieldworkresults}. The purpose of this fieldwork is to determine geohydrological subsurface parameters, transmissivity ($T$) and storativity ($S$), which are used as input for further investigation into upscaling these systems. 

This chapter contains the analysis of gathered fieldwork data. First, the methodology for data analysis, including some theoretical background, is explained (section \ref{section:derivation_methods}). Section \ref{section:TS} contains the derivation of the local geohydrological parameter values: $T$ and $S$. Finally, the chapter concludes with the determination of parameter bandwidths (section \ref{section:fieldwork_results}) which will be used in further model simulations to investigate the upscaling of ASR systems in northern Ghana. 

\section{Parameter derivation methods}
\label{section:derivation_methods}

\subsection{Theoretical model definition}
In large parts of the northern Ghana the geohydrological soil characteristics are unknown. Strong variations in the local geological conditions make it necessary to obtain information about local soil stratification. The most reliable site-specific information was recorded during the drilling of boreholes (2016). The borehole log-sheets (appendix \ref{chapter:Borehole_logsheets}) are used as a starting point for the construction of the theoretical models. 
\\

Despite differences in soil types, the research locations show similarities in stratification. In each case the upper 50 meters is divided into two or three layers, consisting of a confining top layer, and below that one or two "aquifers". Groundwater tables are predominantly positioned in the first aquifer. Based on these observations three simplified theoretical models for the analysis of fieldwork data are derived, as depicted in figure \ref{fig:schematic_fieldwork_analysis}. 

\begin{figure}[h!]
	\centering
	\begin{subfigure}[b]{0.21\linewidth}
		\centering\includegraphics[width=0.6\linewidth]{Schematic_general_analysis}
		\captionsetup{justification=centering}		
		\caption{\label{fig:Schematic_general_analysis}}
		\end{subfigure}%\hfill
	\begin{subfigure}[b]{0.12\linewidth}
		\centering\includegraphics[width=0.6\linewidth]{arrow_right}
		\end{subfigure}%\hfill
		%{\LARGE$\yrightarrow{}$}
	\begin{subfigure}[b]{0.21\linewidth}
		\centering\includegraphics[width=0.6\linewidth]{Schematic_1lay_analysis}
		\captionsetup{justification=centering}		
		\caption{\label{fig:Schematic_1lay_analysis}}
		\end{subfigure}%\hfill
	\begin{subfigure}[b]{0.21\linewidth}
        \centering\includegraphics[width=0.6\linewidth]{Schematic_2lay_analysis}
		\captionsetup{justification=centering}		
		\caption{\label{fig:Schematic_2lay_analysis}}
		\end{subfigure}
	\begin{subfigure}[b]{0.21\linewidth}
        \centering\includegraphics[width=0.6\linewidth]{Schematic_3lay_analysis}
		\captionsetup{justification=centering}		
		\caption{\label{fig:Schematic_3lay_analysis}}
		\end{subfigure}
	\captionsetup{justification=centering}	
	\caption{Schematic cross-sectional view of (\subref{fig:Schematic_general_analysis}) generalized northern Ghana soil stratification and simplified representations: (\subref{fig:Schematic_1lay_analysis}) a single layer system, ~(\subref{fig:Schematic_2lay_analysis}) a double layer system, and ~(\subref{fig:Schematic_3lay_analysis}) a system with two layers and partial penetration of the well} 
	\label{fig:schematic_fieldwork_analysis}
\end{figure} 

These simplified models (\ref{fig:Schematic_1lay_analysis} - \ref{fig:Schematic_3lay_analysis}) mimic local conditions, making the derivation of representative hydraulic subsurface characteristics (T and S) possible (Kruseman \& de Ridder, 2000). Adding more layers provides more degrees of freedom which potentially leads to a more accurate simulations with double layered models. However, a maximum of two soil layers are implemented to limit chances of equifinality. \\ 

\subsection{Techniques in analysis}
\label{section:techniques_analysis}
An increase in the number of layers requires more complex methods for the derivation of parameters. This section contains a detailed description of techniques applied for the derivation of (multiple) hydraulic groundwater parameters. \\


\textbf{Theis's method} \\ 
Groundwater drawdown due to the withdrawal of water can be determined analytically with Theis's equation (Equation \ref{eq:theis}). Theis's method is applicable on the situation depicted in \ref{fig:Schematic_1lay_analysis}; a constant rate pumping test in a fully penetrating well in a confined single layer aquifer (Kruseman \& de Ridder, 2000). The analytical solution is suitable for obtaining a first indication of geohydrological parameters.   

\begin{equation}
\label{eq:theis}
 s = \frac{Q}{4\pi K D} exp1(u)
\end{equation}

\begin{equation}
 u = \frac{r^{2} S}{4 K D t}
\end{equation}

Where $s$ (m) is the drawdown at distance $r$ (m) from the well, $Q$ (m$^{3}$) is the constant well discharge , $KD$ (m$^{2}$/d) is the aquifer transmissivity ($KD$ = $T$), $S$ (-) is the aquifer storativity, $t$ (d) is the time measured from the start of pumping and $exp1$ is the exponential integral. The drawdown measurements in this research are limited to in-well measurements. The distance $r$ in Theis's equation is assumed to be the length of the well radius (0.0635 m). Theis's method is applicable for the time of pumping as well as the recovery process. The script below shows the implementation of Theis's method in Python.\\

\begin{python}[h!]
def drawdown(t, T, S):
    s = Qo / (4 * np.pi * T) * exp1(ro ** 2 * S / (4 * T *t))
    s[t > toff] -= Qo / (4 * np.pi * T) * exp1(ro ** 2 * S / (4 * T *(t[t>toff] - toff)))   
    return s
\end{python}

\textbf{Analytic Element Modelling in TTim}\\
TTim is a computer program based on analytic elements and designed for the analysis of transient groundwater flow in one or more layers. Multiple elements (and types of elements) can be added to specific predefined model layers. The use of TTim makes it possible to take additional well characteristics into account. Groundwater heads can be determined inside the well and the model optionally accounts for borehole storage and well skin resistance. Well discharge can be toggled on and off multiple times. This allows simulations of both  single pumping and recovery tests and long term well operation. \\

The analysis of the pumping and recovery tests is performed with the TTim Model3D configuration. The inclusion of a single well element is sufficient in this case. Depending on which subsurface model is used (\ref{fig:schematic_fieldwork_analysis}) the well (analytic element) is screened in one or more model layers. The top layer is configured as phreatic layer, meaning the top layer storage coefficient ($S$) is a phreatic storage coefficient ($S_y$). This is based on observed initial groundwater tables, which .... . Multiplying this value with the aquifer thickness is therefore no longer needed. Each layer in the model has a thickness of 1 meter. This means derived hydraulic conductivities ($k$) can be interpreted as transmissivities ($T$) and the storage is expressed as the layer storage coefficient ($S$). This is done to directly derive $T$ and $S$ values. Additionally, this approach automatically corrects for the unknown thickness of the deepest soil layer in which the well is screened. There is no information about soil conditions beyond the bottom of the wells in the borehole log-sheets (\ref{chapter:Borehole_logsheets}). \\

\textbf{MODFLOW}\\
The modelling of upscaling scenario's (see chapter \ref{chapter:model_scenarios}) is done with MODFLOW, a finite difference model for groundwater flow developed by the USGS. MODFLOW is the international standard in groundwater simulation (bron??). MODFLOW is not used for the derivation of geohydrological parameters. Optimal parameters derived with TTim models are implemented in corresponding MODFLOW models to validate the results obtained with TTim.

\subsection{Optimization functions}
Pumping test data (section \ref{section:fieldwork_results}) is used as input for the derivation of local geohydrological parameter values. The values of $T$ and $S$ are determined by fitting the analytical solutions and TTim models to the data. In curve fitting process two optimization functions are used. \\

\textbf{Fmin-RMSE optimization}\\
Differences between the measured and modelled drawdown curves can be expressed by the RMSE-value, equation \ref{eq:RMSE} (bron??). The Fmin function is applied (part of Python's scipy.optimize package) to minimize the difference between modelled and observed drawdowns. This optimization results in optimal $T$ and $S$ values (and optionally values for borehole storage and well skin resistance) that represent local conditions. A Python implementatio of Fmin optimization is given below. The example shows an optimization of five parameters ($T$ and $S$ values for two model layers and well skin resistance). \\ 

\begin{equation}
\label{eq:RMSE}
 RMSE = \sqrt{\Sigma\frac{(s_{mod}-s_{field})^{2}}{N}}
\end{equation}

Where $s_{\text{mod}}$ is the modelled drawdown (m), $s_{\text{field}}$ is the observed drawdown (m) and N is the number of data points. \\
 
\begin{python}[h!]
def optimTTim_Qvar(params, t, meas):
    kaq = np.zeros(2)
    Saq = np.zeros(2)
    kaq[0] = params[0]             
    kaq[1] = params[1]
    Saq[0] = params[2]
    Saq[1] = params[3]
    res = params[4]
    s = drawdownTTim_Qvar(t, kaq, Saq, res)
    error = np.sqrt(np.mean((s-meas)**2)) 
    return error

xopt = fmin(optimTTim_Qvar, x0=[10, 10, .01, .001, 0.1], args=(to[mask], do[mask]), xtol=1e-4)
\end{python}
\bigskip

\textbf{Calibration function}\\
TTim also has an in-built calibration function. This optimization function is also applied to improve robustness of the parameter value derivation. In the Python snippet below, an example of the TTim Calibrate function is given. It is the same example mentioned in the Fmin optimization above.\\ 

\begin{python}[h!]
cal = Calibrate(mlc)
cal.parameter(name='kaq0', layer=0, initial=10, pmin=0)
cal.parameter(name='kaq1', layer=1, initial=10, pmin=0)
cal.parameter(name='Saq0', layer=0, initial=.01, pmin=0, pmax=0.3)
cal.parameter(name='Saq1', layer=1, initial=.001, pmin=0, pmax=0.3)
cal.parameter(name='res', par=wc.res, initial=0.1)
cal.series(name='obs3', x=ro, y=0, layer=[0,1], t=to[mask], h=-do[mask])
cal.fit()
\end{python}

These optimization methods require an initial estimate for the parameters. It is possible there is more than one optimal solution which makes the outcome of the optimization dependent on the choice of initial values. Other studies found that $T$ and $S$ values are commonly low in northern Ghana (Owusu et al, 2017). Based on these other studies the following initial conditions are applied: $k_{aq0}$ is 10 (m/d), $k_{aq1}$ is 10 (m/d), $S_{aq0}$ is 0.01 (-), $S_{aq1}$ is 0.001 (-) and well resistance is 0.1 (d). The actual well radius is used as the (initial) borehole storage: 0.0635 (m). Boundary conditions are applied to avoid the optimization resulting in physically improbable parameter values, i.e. negative parameter values and unnaturally high storativity values (greater than 0.3 (-)).\

%(write something over initial conditions. Such small values. Not one single best solution. multiple 'best' solutions potentially close to each other. so solutions highly influential by the arbitrary chosen initial conditions. For each location several attempts done to see which initial conditions score pretty good. And subsequently generalization of those initial parameters applied per location/pumping test. for example better fit at Bingo at initial condition for T (KD) of 10, 10 (lay one and two) for fmin then for cal (2 layered system.) But 2 layered system all of sudden scores better with initial conditions 5, 1 for example.  

\section{From fieldwork data to $T$ \& $S$ values}
\label{section:TS}
The methods and models mentioned in the previous section are applied on the measurements from the five locations: Bingo, Nungo, Nyong Nayili, Janga and Ziong. A complete overview of all simulations can be found in appendix \ref{chapter:Extense_fieldwork_analaysis}. The most important outcomes of this analysis are discussed below for each of the five locations.

\subsection{Location: Bingo}

\textbf{Site inspection}\\
Sloping landscape, some rocks at surface. Area often struck by bush fires, charred vegetation abundant, agricultural fields not ready for use. Although Volta river not far (on map), no river or ponds seen in direct surroundings. Wet season flooding caused by rain and ’popping up’ out ground, labelled as height (1-2m) and disappears in a few days. Well located at walking distance from nearest community. Steel lid, no well screen perforations observed above surface level. \\

\textbf{Measurement quality}\\
Start delayed due to malfunctioning power converter. Tangled rope: Position lowest diver undesirably high and hand measurements not completely possible, result: big gap in data. Recovery test started at an early stage. \\

\textbf{Fit analysis} \\
Data points are missing. Total drawdown unknown from a certain moment in time. The analytical (Theis) solution is not capable of fitting the data. This is most definitely not the case for the data analysed by the use of TTim. The use of both parameter optimization methods are capable of dealing with the lack of data. Both optimizations find optimal parameter values at which drawdown curve exceeds lowest measured groundwater levels. By this example it is shown it is not by definition required to feature complete drawdown curves during parameter determination. Even with incomplete drawdown time series it is possible to obtain reasonable fits. Complete overview of all the optimizations applied (13 optimizations in all) can be found in appendix.. tell about the wobbly curve, due to measured fluctuation in discharge. \\

\begin{figure}[h!]
 \centering
 \includegraphics[width=\linewidth]{bingo_multi_lay_best}
 \captionsetup{justification=centering} 
 \caption{Bingo - multi-layer best fits}
 \label{fig:Bingo_best}
\end{figure}

\begin{table}[h!]
\small
\centering
\caption{Bingo - overview best fit parameters}
\label{tab:bing_table}
\begin{tabular}{l|l|l|l|ll|ll|l}
\hline 
\textbf{}       & \textbf{Method} & \textbf{Stor [m]} & \textbf{Res [d]} & \textbf{T1}  & \textbf{T2   [m$^2$/d]}  & \textbf{S1}  & \textbf{S2 [-]}  & \textbf{RMSE [m]} \\ \hline \hline
Analytical                & fmin             & -             & -            & 10.83      & -          & 2.0e-04    & -          & 0.798133 \\
1 lay                     & fmin             & 0.0647        & 5.6e-02      & 26.23      & -          & 6.6e-03    & -          & 0.162757 \\
2 lay                     & fmin             & 0.0635        & -            & 2.8e-04    & 08.25      & 3.0e-03    & 2.1e-06    & 0.107380 \\
2 lay (pp)                & fmin             & 0.0597        & -            & 8.6e-04    & 07.44      & 7.1e-03    & 6.3e-06    & 0.078188 \\ \hline    
\end{tabular}
\end{table}


%\begin{figure}[h!]
%	\centering
%	\begin{subfigure}[b]{1\linewidth}
%		\centering\includegraphics[width=1\linewidth]{bingo_1lay_analysis}
%		\captionsetup{justification=centering}		
%		\caption{\label{fig:bingo_1lay_analysis}}
%		\end{subfigure} \\ %\hfill
%	\begin{subfigure}[b]{1\linewidth}
%        \centering\includegraphics[width=1\linewidth]{bingo_2lay_analysis}
%		\captionsetup{justification=centering}		
%		\caption{\label{fig:bingo_2lay_analysis}}
%		\end{subfigure} \\
%	\begin{subfigure}[b]{1\linewidth}
%        \centering\includegraphics[width=1\linewidth]{bingo_3lay_analysis}
%		\captionsetup{justification=centering}		
%		\caption{\label{fig:bingo_3lay_analysis}}
%		\end{subfigure}
%	\captionsetup{justification=centering}	
%	\caption{Pumping test fit TS results: (\subref{fig:bingo_1lay_analysis}) single layered, ~(\subref{fig:bingo_2lay_analysis}) double layered and ~(\subref{fig:bingo_3lay_analysis}) triple layered (partially penetrating)} 
%	\label{fig:bingo_fieldwork_analysis}
%\end{figure} 

%
%\begin{table}[h!]
%\small
%\centering
%\caption{Bingo - overview best fit parameters}
%\label{tab:bingo_table}
%\begin{tabular}{l|l|l|lll|lll|l}
%\hline 
%\textbf{}       & \textbf{Stor [m]} & \textbf{Res [d]} & \textbf{T1}& \textbf{T2}  & \textbf{T3   [m$^2$/d]}  & \textbf{S1}& \textbf{S2}  & \textbf{S3 [-]}  & \textbf{RMSE [m]} \\ \hline
%\textbf{Single Lay}       & \textbf{} & \textbf{} & \textbf{}& \textbf{}& \textbf{}  & \textbf{}& \textbf{}& \textbf{}  & \textbf{}                    \\ \hline
%Analytical                & -             & -            & 10.83      & -          & -          & 2.0E-04    & -          & -          & 0.798         \\
%Curve fit                 & -             & 0.05         & 25.52      & -          & -          & 7.3E-05    & -          & -          & 0.166         \\
%Cal                       & -             & 2.5E-04      & 21.94      & -          & -          & 5.9E-19    & -          & -          & 0.175         \\
%{\textbf{}}               &               &              &            &            &            &            &            &            &               \\ 
%\textbf{Double Lay}       & \textbf{} & \textbf{} & \textbf{}& \textbf{}& \textbf{}  & \textbf{}& \textbf{}& \textbf{}  & \textbf{}                    \\ \hline
%Curve fit                 & -             & 0.06         & 22.88      & 0.40       & -          & 4.2E-04    & 8.0E-04    & -          & 0.167         \\
%Cal                       & -             & 0.02         & 11.63      & 0.57       & -          & 3.3E-07    & 4.7E-06    & -          & 0.413         \\ {\textbf{}}               &               &              &            &            &            &            &            &            &               \\ 
%\textbf{Triple Lay}       & \textbf{} & \textbf{} & \textbf{}& \textbf{}& \textbf{}& \textbf{}& \textbf{}& \textbf{}& \textbf{}                        \\ \hline
%Curve fit                 & -             & -            & 6.28       & 1.6E-03    & 0.86       & 1.8E-06    & 1.7E-02    & 2.0E-03    & 0.163         \\
%Cal                       & -             & -            & 17.95      & 3.76       & 3.35       & 0.18167    & 0.29988    & 0.11243    & 0.076         \\ \hline    
%\end{tabular}
%\end{table}


\textbf{Final remark}
korte uitleg welke fit nou het beste is? en waarom. 

\subsection{Location: Nungo}

\textbf{Site inspection}

\textbf{Measurement quality}

\textbf{Fit analysis}

\textbf{Final remark}

\subsection{Location: Nyong Nayili}


\textbf{Site inspection}

\textbf{Measurement quality}
data skip applied. 
\textbf{Fit analysis}

\begin{figure}[h!]
 \centering
 \includegraphics[width=\linewidth]{Nyong_Nayili_multi_lay_best}
 \captionsetup{justification=centering} 
 \caption{Nyong Nayili - multi-layer best fits}
 \label{fig:Nyong_Nayili_best}
\end{figure}

\begin{table}[h!]
\small
\centering
\caption{Nyong Nayili - overview best fit parameters}
\label{tab:Nyong_Nayili_table}
\begin{tabular}{l|l|l|l|ll|ll|l}
\hline 
\textbf{}       & \textbf{Method} & \textbf{Stor [m]} & \textbf{Res [d]} & \textbf{T1}  & \textbf{T2   [m$^2$/d]}  & \textbf{S1}  & \textbf{S2 [-]}  & \textbf{RMSE [m]} \\ \hline \hline
Analytical                & fmin             & -             & -            & 06.00      & -          & 3.0e-01    & -          & 0.751699 \\
1 lay                     & cal              & 0.2419        & -            & 13.35      & -          & 7.8e-05    & -          & 0.457474 \\
2 lay                     & cal              & 0.2436        & -            & 06.95      & 06.98      & 4.6e-06    & 3.6e-05    & 0.456774 \\
2 lay (pp)                & fmin             & 0.2659        & 1.7e-02      & 1.7e-04    & 28.61      & 1.1e-02    & 4.4e-06    & 0.450121 \\ \hline    
\end{tabular}
\end{table}


\textbf{Final remark}


\subsection{Location: Janga (1/2)}

\textbf{Site inspection}

\textbf{Measurement quality}

\textbf{Fit analysis}

\begin{figure}[h!]
 \centering
 \includegraphics[width=\linewidth]{Janga1_multi_lay_best}
 \captionsetup{justification=centering} 
 \caption{Janga first attempt - multi-layer best fits}
 \label{fig:Janga1_best}
\end{figure}

\begin{table}[h!]
\small
\centering
\caption{Janga first attempt - overview best fit parameters}
\label{tab:Janga1_table}
\begin{tabular}{l|l|l|l|ll|ll|l}
\hline 
\textbf{}       & \textbf{Method} & \textbf{Stor [m]} & \textbf{Res [d]} & \textbf{T1}  & \textbf{T2   [m$^2$/d]}  & \textbf{S1}  & \textbf{S2 [-]}  & \textbf{RMSE [m]} \\ \hline \hline
Analytical                & fmin             & -             & -            & 08.84      & -          & 3.0e-01    & -          & 1.338604 \\
1 lay                     & fmin             & 0.0635        & -9.7e-03     & 09.09      & -          & 1.6e-02    & -          & 1.382181 \\
2 lay                     & fmin             & 0.1287        & -            & 12.48      & 1.3e-04    & 1.9e-02    & 1.1e-08    & 1.444546 \\
2 lay (pp)                & fmin             & 0.0635        & -            & 9.1e-05    & 15.19      & 4.3e-08    & 3.1e-03    & 1.530254 \\ \hline    
\end{tabular}
\end{table}

\textbf{Final remark}

\subsection{Location: Janga (2/2)}

\textbf{Measurement quality}

\textbf{Fit analysis}

\begin{figure}[h!]
 \centering
 \includegraphics[width=\linewidth]{Janga2_multi_lay_best}
 \captionsetup{justification=centering} 
 \caption{Janga second attempt - multi-layer best fits}
 \label{fig:Janga2_best}
\end{figure}

\begin{table}[h!]
\small
\centering
\caption{Janga second attempt - overview best fit parameters}
\label{tab:Janga2_table}
\begin{tabular}{l|l|l|l|ll|ll|l}
\hline 
\textbf{}       & \textbf{Method} & \textbf{Stor [m]} & \textbf{Res [d]} & \textbf{T1}  & \textbf{T2   [m$^2$/d]}  & \textbf{S1}  & \textbf{S2 [-]}  & \textbf{RMSE [m]} \\ \hline \hline
Analytical                & fmin             & -             & -            & 15.97      & -          & 3.0e-01    & -          & 0.570855 \\
1 lay                     & fmin             & 5.4e-07       & -9.7e-03     & 13.54      & -          & 1.9e-02    & -          & 0.550853 \\
2 lay                     & fmin             & 0.2228        & -2.2e-02     & 02.05      & 08.13      & 2.1e-02    & 4.1e-04    & 0.544680 \\
2 lay (pp)                & fmin             & 0.2005        & -3.1e-02     & 06.59      & 00.86      & 9.4e-05    & 2.1e-03    & 0.544540 \\ \hline    
\end{tabular}
\end{table}

\textbf{Final remark}

\subsection{Location: Ziong (monitoring)}


\textbf{Site inspection}

\textbf{Measurement quality}
multi day use. Anaytical solution not applied. 
\textbf{Fit analysis}

\begin{figure}[h!]
 \centering
 \includegraphics[width=\linewidth]{Ziong_multi_lay_best}
 \captionsetup{justification=centering} 
 \caption{Ziong - multi-layer best fits}
 \label{fig:Ziong_best}
\end{figure}

\begin{table}[h!]
\small
\centering
\caption{Ziong - overview best fit parameters}
\label{tab:Ziong_table}
\begin{tabular}{l|l|l|l|ll|ll|l}
\hline 
\textbf{}       & \textbf{Method} & \textbf{Stor [m]} & \textbf{Res [d]} & \textbf{T1}  & \textbf{T2   [m$^2$/d]}  & \textbf{S1}  & \textbf{S2 [-]}  & \textbf{RMSE [m]} \\ \hline \hline
1 lay                     & fmin             & 0.0382        & -            & 01.76      & -          & 1.1e-03    & -          & 0.254574 \\
2 lay                     & fmin             & 0.0635        & -0.05        & 00.38      & 01.05      & 2.9e-02    & 1.2e-03    & 0.240162 \\
2 lay (pp)                & fmin             & 0.0147        & -0.08        & 00.23      & 00.78      & 2.6e-02    & 1.3e-03    & 0.243108 \\ \hline    
\end{tabular}
\end{table}

\textbf{Final remark}

\section{Theoretical validation}

\textbf{Soil analysis}

\textbf{VES analysis}


\section{Fieldwork results}
\label{section:fieldwork_results}

summary of the main results
Recommendations for further investigation if applicable. 

Waardes komen overeen (zelfde range/ordegrootte) met theorie. zoek dit even na \\

Er is geen duidelijk onderscheid of 1, 2 dan wel 3 laags (partially penetrating) beter dan wel slechter scoort. dus alle schematische gelaagde bodemoppbouw mogelijk. \\ 

bespreek hier de tekortkoming van het meten in de well zelf als enige. Werkt gewoon niet heel fijn of nauwkeurig. Voor indicatie wel goed. \\ 

vertel over het resultaat van de twee vormen van analyse. fmin versus calibration function. Toch een verschil in alhoritme (numerieke solver). Maar resultaten beide slechts tot op zekere hoogte nauwkeurig. Maar dit zal eerder komen door de metingen zelf (in borehole) en de omstansigheden van northern Ghana. \\ 

does not happen to often. but if T or S values not in right 'order', it is alsways assumed highest T values in layer under confining layer (not in lowest layer). based on the known soil ype of this layer in borehole logsheet (appendiox..). \\

schrijf appendix over de lambda bepaling. 